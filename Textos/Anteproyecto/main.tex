% Anteproyecto: Capa de aguabilidad en la subsuperficie
% Plantilla para Overleaf (LaTeX)
\documentclass[12pt,a4paper]{article}
\usepackage[utf8]{inputenc}
\usepackage[T1]{fontenc}
\usepackage[spanish]{babel}
\usepackage{geometry}
\usepackage{longtable}
\usepackage{booktabs}
\usepackage{array}
\usepackage{multirow}
\usepackage{hyperref}
\usepackage{amsmath}
\usepackage{enumitem}
\usepackage{datetime2}
\usepackage{natbib}
\setcitestyle{breaklines}

\geometry{left=25mm,right=25mm,top=25mm,bottom=25mm}

\begin{document}

% ----------------- Portada -----------------
\begin{center}
  \LARGE\textbf{Instituto de Física}\\[6pt]
  \Large\textbf{Propuesta de Trabajo de Grado}\\[24pt]
\end{center}

\vspace{6pt}

\noindent\textbf{Nombre del Estudiante:} Santiago Andres Orjuela Montealegre \\[4pt]
\noindent\textbf{Número de identificación:} 1006508118 \\[6pt]
\noindent\textbf{Asesor:} Jorge Ivan Zuluaga Callejas \\[6pt]
\noindent\textbf{Modalidad:} Investigación \\
\noindent\textbf{Ciudad, Fecha:} Medellín 2025/10/18 \\[12pt]

\noindent\textbf{Título del Proyecto:}\\
\begin{center}
  \Large\textbf{Capa de Aguabilidad en la Subsuperficie Planetaria}
\end{center}

\vspace{12pt}

% ----------------- Justificación -----------------
\section*{Justificación del proyecto}
El estudio de la distribución térmica y composicional en el interior terrestre constituye la base para comprender su evolución geodinámica y su capacidad para albergar agua líquida en profundidad. Diversas investigaciones han demostrado que la temperatura, la presión y la composición mineralógica controlan la estabilidad del agua y de las fases hidratadas dentro de la litosfera y el manto superior \cite{stacey2008physics}. Estas variables determinan la existencia de lo que podría denominarse una \textit{capa de aguabilidad}, es decir, una franja del interior planetario donde las condiciones físico-químicas permiten la persistencia de agua líquida o de minerales hidratados.

En la Tierra, la caracterización de esta capa es esencial para comprender procesos como la generación del flujo de calor, la deshidratación de minerales en zonas de subducción, la distribución de vida microbiana en ambientes profundos y la dinámica del ciclo hidrológico interno. Modelos recientes de geotermia continental muestran que la producción radiactiva y la conductividad térmica controlan fuertemente el gradiente de temperatura en la litosfera y, por tanto, la profundidad máxima a la que puede mantenerse agua en estado líquido \cite{hasterok2011heat}.

Comprender la ``aguabilidad'' del subsuelo terrestre también tiene implicaciones astrofísicas. La estructura térmica de la Tierra sirve como modelo para estimar la presencia de capas acuosas en exoplanetas rocosos, cuyo potencial de habitabilidad depende del equilibrio entre el flujo de calor interno, la composición mineral y la retención de agua en el manto \cite{unterborn2016scaling}. La extrapolación de modelos termodinámicos bien calibrados a otras masas planetarias permite evaluar hasta qué punto los procesos de hidratación y deshidratación podrían reproducirse en mundos similares a la Tierra, condicionando la existencia de reservorios internos de agua y, en última instancia, la posibilidad de ambientes habitables bajo la superficie.

En este contexto, el presente proyecto busca integrar modelos térmicos del interior terrestre con ecuaciones de estado y datos de conductividad mineral para delimitar la capa de aguabilidad tanto en la Tierra como en planetas rocosos análogos. Esta aproximación, que combina geofísica, termodinámica y astrofísica planetaria, contribuirá a establecer un marco físico que relacione la estructura térmica interior con la habitabilidad potencial de los mundos terrestres.
\noindent

% ----------------- Marco conceptual de habitabilidad -----------------
\section*{Introducción y marco de habitabilidad}
\label{sec:marco-habitabilidad}
La definición clásica de habitabilidad planetaria suele estar anclada en la \emph{zona de habitabilidad circunestelar} (ZH), el anillo en torno a una estrella dentro del cual un planeta con atmósfera adecuada puede mantener agua líquida \emph{en superficie} durante escalas de tiempo geológicas \citep{kasting1993habitable,kopparapu2013habitable}. Bajo esta óptica, la existencia de agua superficial se adopta como un proxy pragmático para la vida basada en química acuosa.

Sin embargo, la habitabilidad no se restringe a la superficie. Múltiples líneas de evidencia astrobiológica y geofísica apuntan a que ambientes \emph{subsuperficiales} pueden albergar agua líquida sostenida por calor interno (p.ej., radiogénico y residual), incluso fuera de la ZH clásica o cuando las condiciones superficiales son hostiles \citep{chyba2005europa}. Esta perspectiva introduce dos conceptos relacionados pero distintos que guían nuestro trabajo:

\begin{itemize}
  \item \textbf{Capa de aguabilidad}: región del interior donde presión y temperatura permiten la \emph{persistencia} de agua líquida (o fases hidratadas) de forma estable.
  \item \textbf{Capa de habitabilidad subsuperficial}: subregión dentro de la capa de aguabilidad que, además, satisface requisitos biofísicos mínimos (rango térmico y de presión compatibles con biomoléculas, disponibilidad de energía geoquímica, y conectividad/porosidad que permita transporte de nutrientes).
\end{itemize}

De este modo, \emph{la existencia de una capa de aguabilidad no implica automáticamente la existencia de una capa habitable}. En este proyecto proponemos cuantificar primero la aguabilidad (controlada por el balance térmico-conductivo y la estructura T–P–$\rho$) y, en un segundo nivel, superponer criterios de habitabilidad para delimitar la fracción realmente apta para vida. Esta comparación con el paradigma de ZH nos permite explorar escenarios en los que la \emph{habitabilidad subsuperficial} persiste incluso cuando la habitabilidad superficial clásica se pierde.

% ----------------- Objetivos -----------------
\section*{Objetivo general}
Analizar la distribución térmica y las condiciones de presión y composición mineralógica del interior terrestre para delimitar la \textit{capa de aguabilidad} en la subsuperficie, y extrapolar dichos resultados a modelos análogos de exoplanetas rocosos, con el fin de establecer los rangos de temperatura y presión que permiten la estabilidad del agua líquida o de fases hidratadas en ambientes internos planetarios.
\noindent

\section*{Objetivos específicos}
\begin{enumerate}[leftmargin=*]
  \item Revisar y sistematizar los modelos de distribución térmica y de producción radiactiva en la litosfera y el manto superior, con base en estudios geotérmicos y geodinámicos previos.
  
  \item Implementar un modelo térmico unidimensional del interior terrestre que integre la conductividad térmica, la producción radiactiva y las propiedades mineralógicas para estimar perfiles de temperatura y presión en la subsuperficie.
  
  \item Determinar los rangos de presión y temperatura que permiten la estabilidad del agua líquida, utilizando ecuaciones de estado y diagramas de fase relevantes para los principales minerales del manto.
  
  \item Analizar la sensibilidad del modelo térmico frente a parámetros como la conductividad térmica, el flujo de calor basal y la composición química, para evaluar la robustez de la capa de aguabilidad definida.
  
  \item Extrapolar los resultados obtenidos a exoplanetas rocosos de diferente masa y composición, comparando la extensión y profundidad relativa de la capa de aguabilidad con respecto a la Tierra.
\end{enumerate}

% ----------------- Metodología -----------------
\section*{Metodología}

La metodología se desarrollará en varias etapas interrelacionadas que permiten abordar progresivamente los objetivos propuestos, combinando análisis bibliográfico, modelado térmico y evaluación termodinámica de la estabilidad del agua en profundidad.

\subsection*{1. Revisión y compilación de información}
Se realizará una revisión exhaustiva de la literatura científica sobre geotermia, estructura térmica y propiedades físicas del interior terrestre, considerando modelos de producción radiactiva \cite{hasterok2011heat}, conductividad térmica mineral \cite{hofmeister1999mantle,hofmeister2005dependence} y ecuaciones de estado relevantes para minerales del manto. Se incluirán estudios de extrapolación planetaria \cite{unterborn2016scaling} y de termodinámica geofísica \cite{stacey2008physics} con el fin de establecer los parámetros iniciales del modelo.

\subsection*{2. Construcción del modelo térmico}

El modelo térmico se formulará siguiendo el procedimiento propuesto por~\cite{hasterok2011heat}, considerando una estructura de conducción estacionaria en una litosfera dividida en capas con distinta producción radiactiva y conductividad térmica efectiva.  
En este enfoque, la temperatura se obtiene mediante soluciones analíticas de la ecuación de conducción de calor unidimensional con fuente interna:

\[
\frac{d^2T}{dz^2} = -\frac{A(z)}{k},
\]

donde \(A(z)\) representa la tasa de producción radiactiva (en W\,m$^{-3}$) y \(k\) la conductividad térmica (en W\,m$^{-1}$\,K$^{-1}$).  

En este anteproyecto adoptaremos una \textbf{parametrización exponencial} de la producción radiactiva cortical, del tipo \(A(z)=A_0 e^{-z/h_r}\), con un fondo radiogénico reducido en el manto litosférico, coherente con los rangos reportados por \cite{hasterok2011heat}. La conductividad térmica se tratará como efectiva por capas (aproximada constante a primer orden), con valores representativos de la literatura experimental \citep{hofmeister1999mantle,hofmeister2005dependence}.

El modelo impone condiciones de frontera de tipo Dirichlet en la superficie (\(T(0)=T_s\)) y de flujo de calor prescrito en la base de la litosfera (\(q_m\)), resolviendo por integración analítica los perfiles de temperatura y flujo en cada capa y garantizando continuidad térmica y de flujo en las interfaces.  

Finalmente, para la extensión del modelo a exoplanetas rocosos, se emplearán las relaciones de escalamiento masa–radio–presión propuestas por \cite{unterborn2016scaling}, junto con parámetros termoelásticos promedio de minerales mayoritarios del manto. Esto permitirá obtener perfiles comparativos de presión, densidad y propiedades térmicas para composiciones análogas o variantes de la Tierra sin detallar aún la implementación computacional específica (reservada para la fase de desarrollo).

\subsection*{2.1. Evolución temporal del flujo interno (conceptual)}
Se considerará una ley de decaimiento efectivo del flujo basal \(q_s(t)\), representativa de la disminución del calor radiogénico y del enfriamiento secular \citep{turcotte2014geodynamics}. De forma cualitativa, un mayor flujo en el pasado implica gradientes térmicos más empinados y capas de aguabilidad potencialmente más delgadas; el enfriamiento secular favorece gradientes más suaves y un posible engrosamiento, sujeto a restricciones de presión y composición.

\subsection*{2.2. Extensión a planetas rocosos análogos (conceptual)}
Se incorporarán relaciones de escalamiento masa–radio–presión \citep{unterborn2016scaling} para extrapolar el marco geotérmico a exoplanetas terrestres. Esto permitirá estimar de manera comparativa la profundidad y el espesor de la capa de aguabilidad y, condicionada por criterios biofísicos, la fracción habitable en mundos de distinta masa y composición.

\subsection*{3. Determinación de la capa de aguabilidad}
Con los perfiles de temperatura y presión obtenidos, se evaluará la estabilidad del agua utilizando ecuaciones de estado y diagramas de fase para el sistema H$_2$O y para minerales hidratados.
La \textit{capa de aguabilidad} se definirá como la región donde las condiciones de presión–temperatura permiten la coexistencia de agua líquida o fases hidratadas estables, identificando su espesor y profundidad promedio.

\subsection*{4. Análisis de sensibilidad y validación}
Se realizará un análisis de sensibilidad del modelo ante variaciones en los parámetros de conductividad térmica, producción radiactiva y flujo de calor basal. Los resultados se compararán con geotermas de referencia y con datos de xenolitos termobarométricos disponibles en la literatura, verificando la consistencia física de los perfiles obtenidos.

\subsection*{5. Extrapolación a exoplanetas rocosos}
Se adaptará el modelo térmico a planetas de diferente masa y composición, utilizando relaciones de escalamiento entre masa, gravedad, radio planetario y presión interna \cite{unterborn2016scaling}. Esto permitirá estimar la profundidad y el grosor de la capa de aguabilidad en exoplanetas terrestres análogos, evaluando su dependencia con la masa planetaria y el contenido radiogénico.

\subsection*{6. Integración y análisis de resultados}
Finalmente, se sintetizarán los resultados comparando los modelos térmicos, las curvas de estabilidad del agua y las escalas planetarias. Se discutirán las implicaciones geofísicas y astrobiológicas de la existencia de una capa de aguabilidad en el interior de la Tierra y de planetas rocosos similares.
\noindent


% ----------------- Antecedentes -----------------
\section*{Antecedentes}

El estudio del régimen térmico de la Tierra y su relación con la estructura interna ha sido abordado desde la geofísica clásica como un problema fundamental de balance de energía. \cite{stacey2008physics} estableció los principios físicos y matemáticos que describen la conducción del calor en el interior terrestre, relacionando la distribución de temperatura con la densidad, la viscosidad y el flujo de calor medido en superficie. Estos fundamentos permitieron desarrollar modelos más precisos sobre la evolución térmica de la litosfera y del manto superior.

Durante las décadas siguientes, la determinación de la conductividad térmica de los minerales constituyentes del manto se convirtió en un aspecto clave para refinar los modelos geotérmicos. Los trabajos experimentales de \cite{hofmeister1999mantle,hofmeister2005dependence} demostraron que la conductividad térmica depende fuertemente de la temperatura, el tamaño de grano y el contenido de hierro, afectando directamente el gradiente térmico y, por tanto, la estimación de la temperatura a grandes profundidades. Estos estudios proporcionaron parámetros realistas para representar la transferencia de calor en materiales del manto y la corteza inferior.

Sobre esta base, \cite{hasterok2011heat} propusieron un modelo integral de producción radiactiva y geotermas continentales que incorpora la estructura composicional de la corteza y el manto litosférico. Su metodología combina datos geofísicos y petrológicos para derivar geotermas por capas, calibradas mediante condiciones de presión–temperatura obtenidas de xenolitos. Este enfoque permitió cuantificar la contribución relativa de la producción radiactiva de la corteza y del manto, así como estimar el espesor térmico de la litosfera en distintas provincias tectónicas. Dicho modelo constituye actualmente una referencia fundamental para estudios de flujo de calor y evolución térmica planetaria.

Más recientemente, la caracterización térmica de planetas rocosos ha cobrado relevancia en el contexto de la astrofísica y la habitabilidad. \cite{unterborn2016scaling} desarrollaron un modelo de escalamiento del interior terrestre a exoplanetas rocosos, considerando cómo la masa y la composición química influyen en la estructura de presión, densidad y temperatura. Este tipo de modelos permite estimar las condiciones internas y la posible existencia de fases acuosas o hidratadas en ambientes subsuperficiales, estableciendo un puente entre la geofísica terrestre y la astrofísica planetaria.

A pesar de los avances, aún persisten vacíos en la delimitación precisa de la franja del interior planetario donde el agua puede mantenerse estable, tanto en la Tierra como en otros planetas. El presente proyecto busca integrar estos aportes en un marco unificado —basado en modelos térmicos y termodinámicos— que permita definir la \textit{capa de aguabilidad} en la subsuperficie terrestre y evaluar su extensión potencial en exoplanetas rocosos análogos.

\noindent

% ----------------- Cronograma -----------------
\section*{Cronograma}


El trabajo se desarrollará en dos fases: la primera durante las 8 semanas restantes del curso \textit{Seminario de Trabajo de Grado}, y la segunda durante las 14 semanas del curso \textit{Trabajo de Grado}. 

\begin{longtable}{p{8cm}p{3cm}p{5cm}}
\toprule
\textbf{Actividad} & \textbf{Duración \t (semanas)} & \textbf{Periodo} \\
\midrule
Revisión bibliográfica y recopilación de parámetros físicos & 4 & Seminario (W1–W4) \\
Formulación del modelo térmico terrestre según Hasterok (2011) & 4 & Seminario (W2–W5) \\
Ajuste de parámetros de conductividad y producción radiactiva (Hofmeister, 1999, 2005) & 3 & Seminario (W4–W6) \\
Determinación preliminar de la capa de aguabilidad en la Tierra & 4 & Seminario (W5–W8) \\
Análisis de sensibilidad y validación del modelo térmico & 3 & Trabajo de Grado (W1–W3) \\
Extrapolación del modelo a exoplanetas rocosos (relaciones de escalamiento) & 6 & Trabajo de Grado (W3–W8) \\
Comparación y análisis de la capa de aguabilidad Tierra–exoplanetas & 5 & Trabajo de Grado (W8–W12) \\
Interpretación final, redacción y revisión con el asesor & 6 & Trabajo de Grado (W9–W14) \\
\bottomrule
\end{longtable}
% Ajusta número de semanas/columnas según duración del semestre.

% ----------------- Requerimientos y presupuesto -----------------
\section*{Requerimientos específicos y presupuesto estimado}

El proyecto no presenta requerimientos específicos

% ----------------- Referencias -----------------
% Usa BibTeX: crea un archivo .bib y compílalo con \bibliography{refs}
% Ejemplo mínimo:
\nocite{*}
\bibliographystyle{apalike}
\bibliography{refs.bib} 


\end{document}
