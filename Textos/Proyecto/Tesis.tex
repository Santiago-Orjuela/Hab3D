% mnras_template.tex 
%
% LaTeX template for creating an MNRAS paper
%
% v3.3 released April 2024
% (version numbers match those of mnras.cls)
%
% Copyright (C) Royal Astronomical Society 2015
% Authors:
% Keith T. Smith (Royal Astronomical Society)

% Change log
%
% v3.3 April 2024
%   Updated \pubyear to print the current year automatically
% v3.2 July 2023
%	Updated guidance on use of amssymb package
% v3.0 May 2015
%    Renamed to match the new package name
%    Version number matches mnras.cls
%    A few minor tweaks to wording
% v1.0 September 2013
%    Beta testing only - never publicly released
%    First version: a simple (ish) template for creating an MNRAS paper

%%%%%%%%%%%%%%%%%%%%%%%%%%%%%%%%%%%%%%%%%%%%%%%%%%
% Basic setup. Most papers should leave these options alone.
\documentclass[fleqn,usenatbib]{mnras}

% MNRAS is set in Times font. If you don't have this installed (most LaTeX
% installations will be fine) or prefer the old Computer Modern fonts, comment
% out the following line
\usepackage{newtxtext,newtxmath}
% Depending on your LaTeX fonts installation, you might get better results with one of these:
%\usepackage{mathptmx}
%\usepackage{txfonts}

% Use vector fonts, so it zooms properly in on-screen viewing software
% Don't change these lines unless you know what you are doing
\usepackage[T1]{fontenc}
\usepackage[spanish]{babel}
%\usepackage[utf8]{inputenc} % LaTeX moderno suele asumir UTF-8 por defecto

% Allow "Thomas van Noord" and "Simon de Laguarde" and alike to be sorted by "N" and "L" etc. in the bibliography.
% Write the name in the bibliography as "\VAN{Noord}{Van}{van} Noord, Thomas"
\DeclareRobustCommand{\VAN}[3]{#2}
\let\VANthebibliography\thebibliography
\def\thebibliography{\DeclareRobustCommand{\VAN}[3]{##3}\VANthebibliography}


%%%%% AUTHORS - PLACE YOUR OWN PACKAGES HERE %%%%%

% Only include extra packages if you really need them. Avoid using amssymb if newtxmath is enabled, as these packages can cause conflicts. newtxmatch covers the same math symbols while producing a consistent Times New Roman font. Common packages are:
\usepackage{graphicx}	% Including figure files
\usepackage{amsmath}	% Advanced maths commands

%%%%%%%%%%%%%%%%%%%%%%%%%%%%%%%%%%%%%%%%%%%%%%%%%%

%%%%% AUTHORS - PLACE YOUR OWN COMMANDS HERE %%%%%

% Please keep new commands to a minimum, and use \newcommand not \def to avoid
% overwriting existing commands. Example:
%\newcommand{\pcm}{\,cm$^{-2}$}	% per cm-squared

% --- Macros útiles ---
\newcommand{\qs}{q_{\mathrm{s}}}
\newcommand{\Az}{A(z)}
\newcommand{\CMF}{\mathrm{CMF}}
\newcommand{\IMF}{\mathrm{IMF}}
\newcommand{\MMF}{\mathrm{MMF}}
\newcommand{\Rearth}{R_{\oplus}}
\newcommand{\Mearth}{M_{\oplus}}

%%%%%%%%%%%%%%%%%%%%%%%%%%%%%%%%%%%%%%%%%%%%%%%%%%

%%%%%%%%%%%%%%%%%%% TITLE PAGE %%%%%%%%%%%%%%%%%%%

% Title of the paper, and the short title which is used in the headers.
% Keep the title short and informative.
\title[Capa de Aguabilidad]{Capa de Aguabilidad en la Subsuperficie Planetaria}

% The list of authors, and the short list which is used in the headers.
% If you need two or more lines of authors, add an extra line using \newauthor
\author[K. T. Smith et al.]{
Keith T. Smith,$^{1}$\thanks{E-mail: publications@ras.ac.uk (KTS)}
A. N. Other,$^{2}$
Third Author$^{2,3}$
and Fourth Author$^{3}$
\\
% List of institutions
$^{1}$Royal Astronomical Society, Burlington House, Piccadilly, London W1J 0BQ, UK\\
$^{2}$Department, Institution, Street Address, City Postal Code, Country\\
$^{3}$Another Department, Different Institution, Street Address, City Postal Code, Country
}

% These dates will be filled out by the publisher
\date{Accepted XXX. Received YYY; in original form ZZZ}

% Prints the current year, for the copyright statements etc. To achieve a fixed year, replace the expression with a number. 
\pubyear{\the\year{}}

% Don't change these lines
\begin{document}
\label{firstpage}
\pagerange{\pageref{firstpage}--\pageref{lastpage}}
\maketitle

% Abstract of the paper
\begin{abstract}
Presentamos un marco físico para definir y cuantificar la \emph{capa de aguabilidad} en el interior de planetas rocosos: la región subsuperficial donde las condiciones de presión y temperatura permiten la persistencia de agua líquida o de fases hidratadas. Combinamos (i) un modelo 1D de geoterma conductivo con conductividad térmica dependiente de composición (promedio Voigt–Reuss–Hill con componentes fonónica y radiativa) y producción radiactiva de decaimiento exponencial en la corteza, (ii) ecuaciones de estado minerales mediante \texttt{BurnMan} para acoplar $T$–$P$–$\rho$, y (iii) un grid de estructuras interiores pre-computadas parametrizado por fracciones de masa del núcleo (CMF) y del hielo (IMF). Contrastamos la \emph{habitabilidad superficial} dada por los límites radiativo–convectivos de la zona de habitabilidad (ZH) con la \emph{habitabilidad subsuperficial}, mostrando que acuíferos profundos pueden existir aun fuera de la ZH clásica cuando el flujo geotérmico es suficiente. Discutimos tendencias del espesor y profundidad de la capa de aguabilidad con $q_s$, composición y masa planetaria, y delineamos criterios preliminares para una capa subsuperficial potencialmente habitable.
\end{abstract}

% Select between one and six entries from the list of approved keywords.
% Don't make up new ones.
\begin{keywords}
planetas y satélites: interiores -- planetas y satélites: planetas terrestres -- astrobiología -- sistemas planetarios -- modelado geotérmico -- habitabilidad
\end{keywords}

%%%%%%%%%%%%%%%%%%%%%%%%%%%%%%%%%%%%%%%%%%%%%%%%%%

%%%%%%%%%%%%%%%%% BODY OF PAPER %%%%%%%%%%%%%%%%%%

\section{Introducción}

La definición clásica de habitabilidad planetaria está anclada en la \emph{zona de habitabilidad} (ZH) circunestelar: el rango de distancias orbitales dentro del cual un planeta rocoso con atmósfera adecuada puede sostener agua líquida en su \emph{superficie} \citep{kasting1993habitable,kopparapu2013habitable}. Este paradigma centrado en la superficie, útil para priorización de objetivos, pasa por alto el potencial de ambientes \emph{subsuperficiales} para albergar dominios acuosos longevos mantenidos por calor interno más que por la irradiación estelar. En el Sistema Solar, posibles reservorios líquidos profundos en Marte y mundos oceánicos cubiertos de hielo (Europa, Encélado) ilustran que el agua puede persistir por fuera de la ZH clásica cuando la conducción o el calentamiento mareal compensan las pérdidas radiativas \citep{chyba2005europa}.

En este trabajo formalizamos y cuantificamos el concepto de \emph{capa de aguabilidad}, definida como el intervalo de profundidad continuo en el interior de un planeta terrestre donde las condiciones de presión ($P$) y temperatura ($T$) permiten la estabilidad termodinámica de agua líquida o de fases minerales hidratadas en escalas geológicas. Diferenciamos esta capa de la \emph{capa de habitabilidad subsuperficial}, más restringida, que además cumple criterios biofísicos (ventana térmica, disponibilidad de fuentes metabólicas de energía y posible porosidad/permeabilidad para transporte de nutrientes).

Avances recientes habilitan una aproximación integrada: (i) grids de estructura interna y evolución térmica parametrizados por fracción de masa del núcleo (CMF) y del hielo (IMF); (ii) física mineral mejorada para conductividad y densidad mediante ecuaciones de estado modernas (\texttt{BurnMan}); y (iii) límites radiativo–convectivos actualizados sobre la habitabilidad superficial \citep{kopparapu2013habitable}. Combinando estos elementos evaluamos cómo la producción y el transporte de calor internos fijan la profundidad, el espesor y la evolución temporal de la capa de aguabilidad en un conjunto de composiciones análogas a la Tierra.

Contribuimos con: (1) una definición físicamente fundamentada de la aguabilidad subsuperficial; (2) un modelo 1D de geoterma conductivo con conductividad térmica dependiente de la composición (promedio Voigt–Reuss–Hill con componentes radiativa y fonónica) y producción radiactiva exponencial \citep{hasterok2011heat}; (3) relaciones de escalamiento para perfiles de presión y profundidades de fronteras en función de masa y composición; y (4) criterios para distinguir regímenes de habitabilidad superficial vs subsuperficial. Nuestro enfoque separa explícitamente los controles externos (estelares) e internos (geotérmicos), cuantificando escenarios donde ambientes acuosos subsuperficiales persisten fuera de la ZH clásica.

La estructura del artículo es la siguiente. La Sección~\ref{sec:theory} desarrolla el marco teórico. La Sección~\ref{sec:methods} describe el pipeline numérico y de datos. La Sección~\ref{sec:results} presenta aplicaciones a análogos del Sistema Solar y a un grid de exoplanetas. La Sección~\ref{sec:discussion} discute implicaciones para la priorización de objetivos. La Sección~\ref{sec:conclusions} resume los hallazgos principales.

A lo largo del texto adoptamos $G$ como la constante gravitacional, $R_\oplus$ y $M_\oplus$ como el radio y la masa terrestres, y empleamos unidades SI salvo indicación en contrario.

\medskip

\section{Marco Teórico}
\label{sec:theory}

\subsection{Definiciones: aguabilidad vs habitabilidad subsuperficial}
Definimos la \emph{capa de aguabilidad} ($\mathcal{L}_\mathrm{aq}$) como el intervalo de profundidad donde puede existir agua líquida (o fases hidratadas estables que actúan como reservorios) dada la trayectoria local de presión–temperatura $(P,T)$. Una \emph{capa subsuperficial habitable} ($\mathcal{L}_\mathrm{hab}$) cumple además: (i) $T_\mathrm{min} \le T \le T_\mathrm{max}$ (p.ej., 273–423 K como rango conservador para actividad microbiana), (ii) porosidad o conectividad de fracturas potenciales y (iii) disponibilidad de gradientes redox (no modelados explícitamente). El concepto de aguabilidad separa la factibilidad termodinámica de la viabilidad biológica.

\subsection{Limitaciones de la zona de habitabilidad clásica}
Los límites de la ZH superficial (\emph{Recent Venus}, \emph{Early Mars}, \emph{moist/runaway greenhouse}, \emph{maximum greenhouse}) derivados de modelos radiativo–convectivos 1D \citep{kopparapu2013habitable,kasting1993habitable} determinan rangos orbitales para agua superficial dados el espectro estelar, el albedo y la composición atmosférica. Sin embargo, los estados térmicos subsuperficiales dependen primariamente del flujo de calor interno $q_s$ y la conductividad térmica $k$, permitiendo que $\mathcal{L}_\mathrm{aq}$ exista más allá de los límites de la ZH o persista tras el colapso atmosférico.

\subsection{Fuentes de calor interno y evolución temporal}
El calor interno planetario proviene de (i) decaimiento radiactivo de larga vida (U, Th, K), (ii) enfriamiento secular del manto y del núcleo y (iii) disipación mareal (no considerada aquí para planetas rocosos aislados). Parametrizamos la disminución temporal del flujo conductivo superficial $q_s(t)$ como $q_s(t)=q_0\exp(t/\tau)$ (tiempo hacia el pasado $t$), consistente con el decaimiento radiactivo agregado \citep{turcotte2014geodynamics}. La producción radiactiva volumétrica $A(z)$ en la corteza sigue a \citet{hasterok2011heat}: $A(z)=A_0 \exp(-z/h_r)$ para $z<z_{\rm Moho}$ y un valor reducido $A_\mathrm{mantle}$ en el manto litosférico.

\subsection{Grid de estructura interior: CMF, IMF, MMF}
Utilizamos modelos de estructura interior precomputados en un espacio de fracción de masa del núcleo (CMF) y del hielo (IMF), con fracción de masa del manto $\mathrm{MMF}=1-\mathrm{CMF}-\mathrm{IMF}$. Estos parámetros gobiernan los perfiles radiales de densidad $\rho(r)$, presión $P(r)$ y gravedad $g(r)$, fijando condiciones de borde para los cálculos geotérmicos y el escalamiento de profundidades de capas.

\subsection{Modelo de geoterma conductivo}
La ecuación conductiva 1D estacionaria con fuentes internas es
\begin{equation}
\frac{\mathrm{d}}{\mathrm{d}z}\left[k(z,T,P)\,\frac{\mathrm{d}T}{\mathrm{d}z}\right] + A(z)=0,
\label{eq:conduction}
\end{equation}
con temperatura de frontera superior $T(0)=T_\mathrm{surf}$ y flujo basal $q_s$. Descomponemos $k = k_\mathrm{lattice} + k_\mathrm{radiative}$; $k_\mathrm{lattice}$ sigue una ley de potencia dependiente de $T$ y $P$, mientras que $k_\mathrm{radiative}$ se activa a $T$ elevadas mediante una transición tipo error. Las soluciones por capas imponen continuidad de $T$ y del flujo en las fronteras composicionales.

\subsection{Física mineral y pipeline de composición}
Las fracciones minerales modales (volumétricas) se convierten a fracciones de masa y luego molares para construir objetos \texttt{Composite} de BurnMan, lo que permite evaluar $\rho(P,T)$ y parámetros elásticos relacionados en cada paso iterativo. La conductividad efectiva adopta el promedio Voigt–Reuss–Hill: $k_\mathrm{VRH}=\tfrac{1}{2}(k_V+k_R)$ con $k_V=\sum f_i k_i$ y $k_R^{-1}=\sum f_i/k_i$.

\subsection{Acoplamiento presión–temperatura y esquema iterativo}
Dado $T_i$ a profundidad $z_i$, estimamos $P_{i+1}$ vía incremento hidrostático $\Delta P = \rho_i g_i \Delta z$, actualizamos el estado del compuesto $(P_{i+1},T_{i+1})$, recomputamos $\rho_{i+1}$ y iteramos hasta cumplir tolerancias $\Delta T < \epsilon_T$, $\Delta P < \epsilon_P$. Así se obtienen perfiles consistentes $(T(z),P(z),\rho(z))$ para delimitar $\mathcal{L}_\mathrm{aq}$.

\subsection{Agua líquida y estabilidad de minerales hidratados}
Aproximamos el campo de estabilidad del agua líquida usando cotas críticas de presión y temperatura (líquido subcrítico: $273 \le T \le 647$ K, con desplazamientos por $P$ elevada). La estabilidad de silicatos hidratados (p.ej., anfíboles, micas) extiende la retención de agua por debajo del campo del líquido puro; sus umbrales de deshidratación aportan límites más profundos para el espesor de $\mathcal{L}_\mathrm{aq}$. El tratamiento completo de equilibrios de fase se deja para trabajo futuro.

\subsection{Escalamiento a exoplanetas}
Las relaciones masa–radio–presión \citep{unterborn2016scaling} ajustan las profundidades de fronteras y los gradientes geotérmicos. Mayor $M$ eleva típicamente $P$ en profundidad y puede estrechar el campo del agua líquida por desplazamientos inducidos por presión, mientras que una CMF alta modifica el perfil gravitacional afectando la eficiencia del transporte de calor en la litosfera.

\subsection{Criterios de habitabilidad subsuperficial}
Adoptamos criterios provisionales: (i) $273 \le T \le 423$ K (envolvente metabólica), (ii) flujo de calor suficiente para mantener agua líquida pero bajo límites esterilizantes ($T<450$ K) y (iii) tiempo de residencia $>10^6$ años (implicando $q_s$ cuasi–estacionario). La disponibilidad de energía redox, la evolución de la porosidad y la circulación de fluidos se reconocen pero no se modelan explícitamente.

\subsection{Novedad y alcance}
Nuestro marco aísla los controles térmicos internos de la irradiación estelar, cuantifica tendencias del espesor de $\mathcal{L}_\mathrm{aq}$ en el espacio CMF/IMF y evalúa cómo planetas más allá de la ZH clásica pueden retener ambientes acuosos subsuperficiales. Esto respalda una priorización refinada de objetivos para futuras búsquedas de bioseñales protegidas de condiciones superficiales hostiles.

\section{Métodos}
\label{sec:methods}
% (To be expanded in subsequent sections.)

\section{Resultados}
\label{sec:results}
% (To be populated once analyses are executed.)

\section{Discusión}
\label{sec:discussion}
% (Interpretation and implications for observations.)

\section{Methods, Observations, Simulations etc.}

Normally the next section describes the techniques the authors used.
It is frequently split into subsections, such as Section~\ref{sec:maths} below.

\subsection{Maths}
\label{sec:maths} % used for referring to this section from elsewhere

Simple mathematics can be inserted into the flow of the text e.g. $2\times3=6$
or $v=220$\,km\,s$^{-1}$, but more complicated expressions should be entered
as a numbered equation:

\begin{equation}
    x=\frac{-b\pm\sqrt{b^2-4ac}}{2a}.
	\label{eq:quadratic}
\end{equation}

Refer back to them as e.g. equation~(\ref{eq:quadratic}).

\subsection{Figures and tables}

Figures and tables should be placed at logical positions in the text. Don't
worry about the exact layout, which will be handled by the publishers.

Figures are referred to as e.g. Fig.~\ref{fig:example_figure}, and tables as
e.g. Table~\ref{tab:example_table}.

% Example figure
\begin{figure}
	% To include a figure from a file named example.*
	% Allowable file formats are eps or ps if compiling using latex
	% or pdf, png, jpg if compiling using pdflatex
	\includegraphics[width=\columnwidth]{example}
    \caption{This is an example figure. Captions appear below each figure.
	Give enough detail for the reader to understand what they're looking at,
	but leave detailed discussion to the main body of the text.}
    \label{fig:example_figure}
\end{figure}

% Example table
\begin{table}
	\centering
	\caption{This is an example table. Captions appear above each table.
	Remember to define the quantities, symbols and units used.}
	\label{tab:example_table}
	\begin{tabular}{lccr} % four columns, alignment for each
		\hline
		A & B & C & D\\
		\hline
		1 & 2 & 3 & 4\\
		2 & 4 & 6 & 8\\
		3 & 5 & 7 & 9\\
		\hline
	\end{tabular}
\end{table}


\section{Conclusions}

The last numbered section should briefly summarise what has been done, and describe
the final conclusions which the authors draw from their work.

\section*{Acknowledgements}

The Acknowledgements section is not numbered. Here you can thank helpful
colleagues, acknowledge funding agencies, telescopes and facilities used etc.
Try to keep it short.

%%%%%%%%%%%%%%%%%%%%%%%%%%%%%%%%%%%%%%%%%%%%%%%%%%
\section*{Data Availability}

 
The inclusion of a Data Availability Statement is a requirement for articles published in MNRAS. Data Availability Statements provide a standardised format for readers to understand the availability of data underlying the research results described in the article. The statement may refer to original data generated in the course of the study or to third-party data analysed in the article. The statement should describe and provide means of access, where possible, by linking to the data or providing the required accession numbers for the relevant databases or DOIs.




%%%%%%%%%%%%%%%%%%%% REFERENCES %%%%%%%%%%%%%%%%%%

% The best way to enter references is to use BibTeX:

\bibliographystyle{mnras}
\bibliography{example} % if your bibtex file is called example.bib


% Alternatively you could enter them by hand, like this:
% This method is tedious and prone to error if you have lots of references
%\begin{thebibliography}{99}
%\bibitem[\protect\citeauthoryear{Author}{2012}]{Author2012}
%Author A.~N., 2013, Journal of Improbable Astronomy, 1, 1
%\bibitem[\protect\citeauthoryear{Others}{2013}]{Others2013}
%Others S., 2012, Journal of Interesting Stuff, 17, 198
%\end{thebibliography}

%%%%%%%%%%%%%%%%%%%%%%%%%%%%%%%%%%%%%%%%%%%%%%%%%%

%%%%%%%%%%%%%%%%% APPENDICES %%%%%%%%%%%%%%%%%%%%%

\appendix

\section{Some extra material}

If you want to present additional material which would interrupt the flow of the main paper,
it can be placed in an Appendix which appears after the list of references.

%%%%%%%%%%%%%%%%%%%%%%%%%%%%%%%%%%%%%%%%%%%%%%%%%%


% Don't change these lines
\bsp	% typesetting comment
\label{lastpage}
\end{document}

% End of mnras_template.tex